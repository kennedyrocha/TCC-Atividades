\documentclass[a4paper,12pt]{article}

\setlength{\parindent}{1.2cm}
\setlength{\parskip}{.2cm}
\setlength{\oddsidemargin}{0.5cm}    % Há um offset obrigatorio de
\setlength{\evensidemargin}{0.0cm}   % 1 inch do lado esquerdo e no
\setlength{\topmargin}{-1.2cm}       % topo da folha
\setlength{\headsep}{1.0cm}
\setlength{\textwidth}{15.5cm}
\setlength{\textheight}{24.2cm}
\renewcommand{\baselinestretch}{1.2}
\renewcommand{\labelitemi}{\tiny{\textbullet}}
\begin{document}

\begin{center}
\textbf{{\LARGE 100 GigaBit Ethernet}} \\ \vspace{0.5cm}
\end{center}

A Ethernet é um conjunto de normas e padrões de rede que define regras numa LAN (Local Internet Network) para a transmissão de dados, implementando o algorítimo CSMA/CD (Carrier Sense Multiple Access with Collision Detection) para acesso a dados com detecção de colisão e o MAC (Medium Access control) para controle de acesso ao meio. Ela surgiu quando Robert Metcalfe escreveu um memorando aos seus chefes descrevendo o potencial dessa tecnologia para redes locais, logo após saiu da empresa onde trabalhava (Xerox), porém conseguiu convencer a DEC (Digital Equitpment Corporation), Intel e Xerox a trabalharem juntas para promoverem a Ethernet como um padrão.

Esse protocolo é atualmente padronizado pelo IEEE 802.3, um grupo de estudo pertencente ao IEEE (Institute of Eletrical and Electronics Engineers), cuja a responsabilidade é estudar e padronizar esse modelo de rede, tal qual atua na camada física e de enlace de dados no modelo OSI (Open Systems Interconnection). Os padrões são especificados por velocidade, ou seja, para cada velocidade há uma normalização.
Dentro da camada física do Modelo OSI, a ethernet define padrões de cabeamento, dispositivos (switches e patch panels) e estruturas para que a velocidade desejada seja atingida.
Já na camada de enlace, é usado um controlador de link lógico para destinar os dados de forma mais eficiente e também o MAC, para que cada dispositivo conectado a rede tenha um endereço único, evitando o envio e processamento desnecessário de informações.
Para interligar essas duas camadas foi desenvolvido o reconciliador e cada norma tem um reconciliador específico, na 100 Gigabit Ethernet é recomendado o CGMII.
Nesse âmbito, a 100 Gigabit, ou 100GE, é um conjunto de normas e tecnologias de rede para transmissão de dados numa velocidade de 100 Gb/s (IEEE Computer Society (2018)).


Nesse padrão, inicialmente é determinada as especificações da camada física (PHY - Physical Layer Device) para a transmissão desses dados, tal qual é divida em subcamadas, são elas: Physical Coding Sublayer (PCS), Forward Error Correction (FEC), Physical Medium Attachment (PMA), Physical Medium Dependent (PMD) e o Medium.

A implementação dessa camada, na 100 Gigabit Ethernet, é dada através do 100GBASE-R ou 100GBASE-X, que é a familia de dispositivos que trabalha na velocidade de 100 Gb/s, sendo que este usa a codificação PCS com codificação (64B/66B), enquanto aquele usa codificação External Sourced (4B/5B, 8B/10B). As demais familias da 100GE (100GBASE-KR4, 100GBASE-CR4, 100GBASE-SR10 e etc.) são baseadas nessas duas codificações, a 100GBASE-R e 100GBASE-X.

A primeira subcamada física (PCS) provê o serviço de codificação/decodificação dos dados em blocos de 66bit (64B/66B), é responsável por distribuir os dados em diferentes faixas, realizar o \textit{scramble} dos blocos de bits, compensação de diferença de taxas entre o reconciliador e o PMA, determinar quando uma conexão foi estabelecida informando então ao gerenciador quando o dispositivo está pronto para uso. 


Já na segunda subcamada física o FEC (Foward Error Correction) age com o objetivo de evitar a perca de dados através da redundância no envio de bits, onde ele faz a mesma adicionando bits ao streaming de dados pelo algorítimo Reed-Salomon, sendo então nomeado como RS-FEC (Reed Solomon Forward Error Correction). Em cada especificação o RS-FEC trabalha de uma forma e, para ser compatível com a 100GE, é necessário que o PMA tenha 4 faixas para envio e outras quatro para recebimento, enfatizando então o requerido na 100GE, que é o modo de operaçao full duplex e oito faixas para transmissão de dados no PMA.

A terceira subcamada, o PMA (Physical Meidium Attachment), fornece o serviço de intermediamento entre um PMA e um cliente, podendo esse cliente ser um PCS, FEC ou outro próprio PMA. Entre esses serviços têm-se a adaptação dos sinais das faixas dos PCS para o número de faixas físicas ou abstratas do cliente, ou seja, ele pode receber 10 faixas de stream de dados e transforma-lá em 4 faixas de stream de dados. Também provê especificações de tempo para transmissão dos dados entre as faixas assim como gerenciamento dos mesmos. O PMA faz o direcionamento de bits de dados para que todos os bits de uma stream vão e voltem pela mesma faixa. Como o PMA possui faixas de transmissão para prover tais serviços, há também uma padronização para tais faixas, onde o 802.3 define que o número de faixas do PMA sempre é divisor do número de faixas do PCS, então para suportar a 100GBASE-R é necessário que o PMA tenha 20 PCSLs. Ainda na segunda camada, quando há a comunicação entre dois PMAs pode-se usar a instanciação CAIU-10 ou CAIU-4, onde a CAIU-10 é o uso de dez faixas a 10.3125 GBd e a CAIU-4 é o uso de quatro faixas a 25.78125 GBd.


A quarta subcamada (PMD) provê o serviço de intermédio entre o PMA e o MDI controlando o envio e recebimento dos dados entre os mesmos, traduzindo o código recebido do PMA de streamings de bit para stramings elétricas ou de streamings de bits para streamings de sinais óticos e o contrário também, onde o PMA trabalha com bits e o MDI com sinais elétricos e/ou óticos. Também na implementação do PMD é decidido qual modo de comunicação/conexão usar, exemplo: Fibra ótica em Single-Mode, MultiMode ou também cabos de cobre. Há maneiras de ser implementado o PMA e alguns dos padronizados pelo 803.2 são os modos: 100GBASE-KR4 que é o PHY a 100 Gb/s, codificação de 64B/66B, RS-FEC e amplitude da modulação de 2 pulsos em 4 faixas de backplane elétrico; o 100GBASE-CR4 com PHY a 100 GB/s, codificação de 64B/66B, RS-FEC e 4 faixas de cabo de cobre balanceado e distância de 5m; o 100GBASE-SR10 com PHY a 100Gb/s, codificação de 64B/66B e 10 Faixas de fibra em multimodo e distância de 100m. Há mais modos padronizados no IEEE Standards for Ethernet (2018) na Tabela 80-1.

Relacionado ao PMD, tem-se ainda o MDI (Medium Dependent Interface), que é a interface de comunicação entre o dispositivo PMD e o Medium, também entendido como meio de comunicação (fibra ótica, cabo de cobre, backplane). Essa interface pode ser compreendida de outro modo como o receptor e/ou transmissor acoplado ao dispositivo PMD, e varia conforme a normativa, sendo que na 100GBASE-SR10 é apresentada três opções de implementação e a recomendada é a que há dez faixas para transmissão e outras dez para recepção (IEEE Standards for Ethernet (2018) - Figura 86-7).

1 - Interface entre o dispositivo e o meio de comunicação

2 - Varia de normativa a normativa

3 - No 100GBASE-SR10 o meio de comunicação é dada através de fibra ótica

4 - MDI na 100GBASE-SR10 pode ser um receptor/transmissor ótico


Corrigir o backplane


\end{document}